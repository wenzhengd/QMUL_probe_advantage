% ======================================================================
% APS / REVTeX 4.2 manuscript (PRA/PRB style)
% Converted from Markdown draft: "Why External Quantum Degrees of Freedom
% Reveal Fluctuational Information Beyond Response"
% ======================================================================

\documentclass[%
  reprint,%
  aps,%
  pra,%
  superscriptaddress,%
  longbibliography,%
  nofootinbib,%
  onecolumn
]{revtex4-2}

% ----------------------------
% Core math + symbols
% ----------------------------
\usepackage{amsmath,amssymb,amsthm,amsfonts}
\usepackage{mathrsfs}
\usepackage{dsfont}
\usepackage{stmaryrd}
\usepackage{mathtools}
\usepackage{bm}
\usepackage{braket}
\usepackage{multirow}

% ----------------------------
% Figures / color / hyperlinks
% ----------------------------
\usepackage{graphicx}
\usepackage{color,xcolor}

\usepackage{hyperref}
\hypersetup{
  colorlinks=true,
  linkcolor=blue,
  citecolor=blue,
  urlcolor=black
}

% ----------------------------
% Circuits + algorithms (optional)
% ----------------------------
\usepackage{qcircuit}
\usepackage{algorithm}
\usepackage{algpseudocode}

% Lists
\usepackage{enumerate}

% ----------------------------
% Common macros
% ----------------------------
\newcommand{\ii}{\mathrm{i}}
\newcommand{\dd}{\mathrm{d}}
\newcommand{\ee}{\mathrm{e}}
\newcommand{\Tr}{\mathrm{Tr}}
\newcommand{\sgn}{\mathrm{sgn}}
\newcommand{\T}{\mathcal{T}}
\newcommand{\Tc}{\mathcal{T}_c}
\newcommand{\Tt}{\tilde{\mathcal{T}}}
\newcommand{\avg}[1]{\left\langle #1 \right\rangle}

\newcommand{\WD}[1]{\textcolor{blue}{#1}}
\newcommand{\AGENT}[1]{\textcolor{red}{AGENTS: #1}}


% Heaviside step function (often used in response theory)
\newcommand{\Heav}{\Theta}

% Bracket notations
\newcommand{\comm}[2]{\left[#1,#2\right]}
\newcommand{\acomm}[2]{\left\{#1,#2\right\}}
\newcommand{\expect}[1]{\langle #1 \rangle}

% ----------------------------------------------------------------------
\begin{document}

\title{Quantum probe advantage of learning many-body systems}
\author{Wenzheng Dong}
\author{Jinzhao Sun}

\affiliation{Queen Mary University of London}

\begin{abstract}
Understanding which physical properties of a quantum many-body system are operationally accessible is a foundational question underlying spectroscopy, thermodynamics, and quantum information science.
Conventional response theory, a cornerstone of many-body physics, perturbs and measures the system itself, yielding susceptibilities constructed exclusively from causally ordered nested commutators.
Here we establish, at the operator level and from first principles, that when auxiliary quantum probe(s) is coherently coupled to a target system and subsequently measured, its reduced dynamics encodes a strictly larger class of information about the system. 
Owing to probe--system entanglement and the resulting non-Markovian backaction, probe observables access multi-time anti-commutator correlators and mixed operator orderings that characterize fluctuations, non-equilibriant behavior, and irreversibility---quantities that are not, in general, determined by response functions alone. We present explicit examples showing how small, controllable quantum probes can efficiently reconstruct physical properties of many-body systems that are otherwise inaccessible or operationally impractical using system-only measurements. 
Our results demonstrate that quantum probes do not merely enhance measurement sensitivity within a closed-system framework, but fundamentally enlarge the algebra of operationally measurable observables, thereby establishing a precise and general notion of quantum probe advantage in many-body physics.
\end{abstract}

\maketitle

% ======================================================================
\section[1]{Introduction}
\WD{Introduction is hard to write: not sure how the story is motivated... to be discussed}

Response theory has long provided a unifying framework for understanding how quantum systems react to external perturbations. In its standard formulation, a system is perturbed by a classical control field and observables of the same system are measured. Linear and nonlinear response functions obtained in this way encode transport coefficients, susceptibilities, and spectral properties.

At the same time, contemporary quantum platforms increasingly employ \emph{quantum probes}: ancillary qubits, spins, cavities, or modes that couple to a target system and are measured directly. Such probe-based protocols underpin quantum noise spectroscopy, decoherence spectroscopy, thermometry, and many forms of open-system characterization. It is often stated that quantum probes ``reveal more information'' than classical probes, but this statement is rarely made precise at the operator level.

The purpose of this work is to give a \emph{structural answer} to the question:
\begin{quote}
\textit{What physical information is inaccessible to system-only response theory but becomes operationally measurable when a quantum probe is introduced?}
\end{quote}
The central result is that the distinction is not merely technological but algebraic. System-only response theory is confined to retarded, commutator-based correlators, whereas probe-based measurements naturally produce Keldysh contour objects whose real-time expansions contain anti-commutators and mixed operator orderings. These additional sectors encode fluctuations, noise, occupations, and irreversibility, which are independent of response functions out of equilibrium.

% ======================================================================

\section[2]{Correlation Fucntions and Operator Orderings}
Response theory is widely used in many-body systems to study crucial physial properties. This is achieved by applying perturbations directly to the system, and measuring operator response thereof.
[\AGENT{start when it is denoted by $M$?}]
Notice that the perturbative Hamiltonians can be thought of using a trivial, or \textit{classical}, probe on the target system. 
\WD{One can assume a probe which after being trace out, left the many-body system-only dynamics gauge-fixed unitary as described in response theory; 
then we can prove that the state of probe is trivially identity. } 

On the other hand, introducing a quantum probe, whose dimension is substantially smaller than the target, will let the reduced dynamics, on target system or probe, non-unitary due to the entanglement between 
them. Notice that the quantum probe is disparate also as the probe's degree of freedom is non-trivial --- the probe's state evoles with the target and the internal entanglement among an ensemble of probes changes. 
The information of the target system will be flown to the probe side and is read out on the probe.    

In this section, we illustrate the different algebraic structures in system-based learning and in probe-based learnng (\WD{learning is inapppropriate ?}), where in the former case, general non-linear respons theory reconstructs nested commutators, and in the latter case, probe-based spectroscopic scheme reconstructs nested anticommutator and commutators altogether.  

\subsection[2_a]{System-based learning: generalized response theory}
% This should be a secton introducing/reviewing the general response theory: start from perturbation to obtained the operator's response to the perturbation.
% all the way -math derivations- to the nested commutators that is at the heart of non-linear response theory. 

\AGENT{This part might be TOO hand-waving: linear response theory is textbook knowledge and is well-known. It is not a good idea to REVIE it in NatPhy. Might need to be more compact to go to GNLR directly. What do you think? }

We briefly review the structure of quantum response theory, emphasizing that classical, system-only probing is fundamentally an estimation problem for causally ordered nested commutators. This algebraic restriction will later serve as a sharp point of contrast with probe-based measurement schemes.

Consider a quantum matter system $M$ with Hamiltonian \(H_0\), subjected to a time-dependent perturbation
\begin{equation}
H(t) = H_0 - \lambda(t)\, M_0(t) ,
\end{equation}
where \(\lambda(t)\) is a real-valued control field and \(M_0(t)\) is a Hermitian system operator. We work in the interaction picture with respect to \(H_0\) (let $\hbar\equiv1$), in which
\begin{equation}
M(t) = e^{i H_0 t} M_0(t) e^{-i H_0 t}.
\end{equation}
The interaction-picture evolution operator is 
\begin{equation}
U_I(T) = \mathcal{T}\exp\!\left(
i\int_{0}^{T} dt\, \lambda(t)\, M(t)
\right),
\end{equation}
where \(\mathcal{T}\) denotes time ordering. The expectation value of an observable \(A\) at the final time \(T\) is
\begin{equation}
\langle A(T) \rangle_\lambda
= \mathrm{Tr}\!\left[
U_I^\dagger(T)\, A(T)\, U_I(T)\, \rho_M
\right],
\end{equation}
with \(\rho_M\) the initial state of $M$ prepared before the perturbation is applied.
Expanding \(U_I(T)\) to first order in \(\lambda\),
\begin{equation}
U_I(T) \simeq 1 + i\int_{0}^{T} dt_1\, \lambda(t_1) M(t_1),
\end{equation}
and retaining only linear terms, the induced change in the expectation value of \(A\) is
\begin{equation}
\delta \langle A(T) \rangle
= i\int_{0}^{T} dt_1\, \lambda(t_1)\,
\langle [A(T), M(t_1)] \rangle .
\end{equation}
This is the Kubo formula\cite{}. The associated retarded susceptibility is
\begin{equation}
\chi_{AM}^R(T,t_1)
= i\,\Theta(T-t_1)\,\langle [A(T), M(t_1)] \rangle,
\end{equation}
which depends solely on a commutator and enforces causality explicitly.


\textbf{General nonlinear response} Higher-order response functions follow directly from the Dyson expansion of \(U_I(T)\). At order \(k\), the contribution to the expectation value takes the universal form
\begin{align}
\delta^{(k)} \langle A(T) \rangle
&= \left(i\right)^k
\int_{0}^{T} dt_1
\int_{0}^{t_1} dt_2
\cdots
\int_{0}^{t_{k-1}} dt_k \,
\lambda(t_1)\lambda(t_2)\cdots\lambda(t_k)
\nonumber\\
&\quad\times
\big\langle
[\cdots[ A(T), M(t_1) ], M(t_2) ], \ldots , M(t_k)]
\big\rangle ,
\end{align}
where the nested integrals impose the causal ordering
\begin{equation}
T \ge t_1 \ge t_2 \ge \cdots \ge t_k ;
\end{equation}
and $\expect{\bullet}\equiv = \Tr[\bullet \rho_M]$.

\textbf{Algebraic structure and interpretation} The key structural result is that \emph{all} orders of classical response theory are built exclusively from causally ordered nested commutators. No anti-commutators or symmetrized correlators appear at any stage of the expansion. 
This restriction follows directly from two assumptions:
(i) the perturbation is direct applied on the system, and
(ii) perturbed system evoles unitarily.

From this perspective, response theory may be viewed as an estimation framework for a specific subset of multi-time operator correlations: the fully retarded nested commutators
\begin{equation}
\big\langle
[\![\![ A(T), M(t_1) ], M(t_2) ], \ldots , M(t_k)]
\big\rangle.
\label{eq:response_correlator}
\end{equation}
To claim and to be explained: These objects encode causal susceptibility and transport properties, but they do not contain independent information about fluctuations, symmetrized noise, or mixed operator orderings. 
Accessing such additional sectors requires going beyond system-only response theory, as we will demonstrate using quantum probes.

\WD{(Not only we need to add refs on GNR / LR theory in conventional many-body physics, but also illustrate if NEW tools -e.g., classical shadow- essentially described by respons theory. Absence of such might weaken our impact.) }

\subsection[2_b]{Quantum probe: open quantum system-based learning and algebraic structure}
% This should be a secton introducing/reviewing the QNS-based learning theory: start from probe-Matter coupling to obtained the probe's response to the perturbation.
% with a few steps, obtained the nested commutator / anticommutator structure as explained in the DISCO;s paper.

The probe-based learning frameworks adopt a perspective fundamentally distinct from system-based response theory, thereof, a quantum probe, or several entangle units as an ensemble, to the matter and infers properties of the matter from the reduced dynamics of the probe.
The probe thus acts as a quantum sensor, and learning is performed by observing probe observables rather than system responses. 
Probe-based learning is also known as quantum noise spectroscopy (QNS) in characterization of quantum devices through estimating correlated (aka non-Markovian) noise spectra.

The essential distinction is algebraic. Classical, system-based response theory is restricted to fully retarded nested commutators, whereas probe-based learning accesses a larger family of operator orderings arising from the open-system evolution of the probe.

We consider a well-controlled probe ensemble \(P\) (\WD{say, it is a $|P|$-qubit system}) coupled to a matter system \(M\). The total Hamiltonian is written as
\begin{equation}
H(t) = H_P(t) + H_M + H_{PM}(t),
\end{equation}
with an effective interaction 
\begin{equation}
H_I(t) = \sum_{p\in {P}} h_p(t)\otimes M_p(t)
\end{equation}
w.r.t. the rotation of $H_P(t)+H_M$ , where \(h_p(t)\) are controlled probe operators (including all unitaries actively applied on probe) and \(M_p(t)\) are Heisenberg-picture matter operators. 
The $M_p(t)$ represent the operator on $M$ (or its subsystem) when the coupling involves subsystem $p$ in $P$. 
Notice that in most Hamiltonian-level open quantum system modeling, for $p\neq p'$, one has orthogonal $P$-operators as $\Tr_P[h_p(t),h_{p'}(t)]\equiv0$ , while $\Tr_M[M_p(t),M_{p'}(t)]\neq0$. 
One example is each $p\in P$ couples to an collective environment $M$, in which case all $M_p$ are equivalent.
The initial state is assumed to be factorized,
\begin{equation}
\rho(0)=\rho_P\otimes\rho_M .
\end{equation}
All experimentally accessible information about the matter enters through probe observables measured at a final time \(T\). 
After tracing out the $M$ degrees of freedom, the probe undergoes open quantum system dynamics governed by multi-time correlation functions of the operators \(M_p(t)\).


Perturbative expansions of the reduced probe dynamics generate $M$ operator strings of multiple time orderings allowed,
\begin{equation}
M_{p_{\pi(1)}}(t_{\pi(1)})\cdots M_{p_{\pi(k)}}(t_{\pi(k)}),
\end{equation}
where the permutations \(\pi\) arise from the forward--backward structure of the probe’s reduced evolution. 
A central result of the DISCO~\cite{} framework by one of us is that all such operator strings can be compactly expressed in terms of a universal basis of nested commutator and anti-commutator structures.


\textit{Nested braketors: general-order algebraic structure}. Define a binary bracket indexed by \(\mu\in\{0,1\}\),
\begin{equation}
\llbracket X,Y \rrbracket_\mu := XY + (-1)^{\mu}YX ,
\end{equation}
so that \(\mu=1\) corresponds to the commutator and \(\mu=0\) to the anti-commutator. 
For \(k\ge 2\), the \((k-1)\)-level nested bracket (or ``braketor'') correlator is defined as
\begin{equation}
\mathcal{M}^{\vec{\mu}}_{\vec{p}}(t_1,\ldots,t_k)
\equiv
\frac{1}{2^{k-1}}
\,\widehat{P}_{{\vec{p}}}
\Big(
\llbracket\cdots \llbracket M_{p_1}(t_1),M_{p_2}(t_2)\rrbracket]_{\mu_1},
M_{p_3}(t_3)\rrbracket_{\mu_2}\cdots,
M_{p_k}(t_k)\rrbracket_{\mu_{k-1}}
\Big),
\label{eq:probe_correlator}
\end{equation}
where \(\vec{\mu}=(\mu_1,\ldots,\mu_{k-1})\in\{0,1\}^{k-1}\) and $\widehat{P}_{{\vec{p}}}$ is an ordering operator ensuring consistent labeling of the operator indices \((p_1,\ldots,p_k)\).

These braketors form a complete operator basis: any admissible \(k\)-point bath operator string appearing in the probe’s reduced dynamics can be expressed as a linear combination of \(\mathcal{M}^{\vec{\mu}}_{\vec{p}}(t_1,\ldots,t_k)\).


\begin{table}[H]
\centering
\begin{tabular}{|c|c|c|c|}
\hline
 & Learnable correlators & $\#$ of combinations & Linear term \\
\hline
\multirow{2}{*}{
\begin{tabular}{c}
\textbf{System-based} \\
(closed system)
\end{tabular}}&
\multirow{2}{*}{$
\mathcal{M}^{\vec{0}}(t_{[k]}) \sim [\cdots[ M_{m_1}(t_1), M_{m_2}(t_2) ], \ldots , M_{m_k}(t_k)] $}
&\multirow{2}{*}{$1$} & $\big[ M_{m_1}(t_1),\, M_{m_2}(t_2) \big]$
\\ &  & & \\
\hline
\multirow{2}{*}{
\begin{tabular}{c}
\textbf{Probe-based} \\
(open system)
\end{tabular}} &
\multirow{2}{*}{$ \mathcal{M}^{\vec{\mu}}(t_{[k]}) \sim \llbracket\cdots \llbracket M_{p_1}(t_1),M_{p_2}(t_2)\rrbracket_{\mu_1},\cdots, M_{p_k}(t_k)\rrbracket_{\mu_{k-1}}$ ~ $\vec{\mu}\in\{0,1\}^{k-1}$}
& \multirow{2}{*}{$2^{k}$} &
$\big[ M_{m_1}(t_1),\, M_{m_2}(t_2) \big]$
\\ & & & 
$\{ M_{m_1}(t_1),\, M_{m_2}(t_2) \}$ \\
\hline
\end{tabular}
\caption{Comparison between system-based (closed-system) and probe-based (open-system) learning of multi-time correlators.
The probe-based approach enlarges the algebra of learnable correlators by accessing both commutator and anti-commutator structures.
The expressions of correlators are adpated from Eq.~\ref{eq:response_correlator} and Eq.~\ref{eq:probe_correlator} for comparison; where subscription $m$ indexes $M$-subsystem.  
The linear term in system-based response theory corresponds to leading non-linear ($k=2$) term in probe-based theory (as $M_{m_1}(t_1)$ therein is not measurement observable). 
}
\label{tab:probe_vs_system_correlators}
\end{table}

\subsection{Algebraic distinction from system-based theory}

From this perspective, the distinction between system-based response theory and probe-based theory becomes precise. 
Classical system-based response theory accesses only the fully retarded sector of this algebra, corresponding to \(\mu_1=\cdots=\mu_{k-1}=1\), yielding nested commutators ordered by causality. 
Probe-based learning, by contrast, accesses linear combinations spanning the full \(\vec{\mu}\in\{0,1\}^{k-1}\) space. 
We will argue that these include anti-commutator and mixed sectors that encode fluctuations, noise, and higher-order correlations. 
Comparison between two schemes are in Table.~\ref{tab:probe_vs_system_correlators}.

Thus, probe-based learning is not an extension of response theory but an open-quantum-system framework whose observables encode a strictly larger and algebraically richer class of multi-time correlators.

Beside the above algebraic distinction, it is crucial to clarify the physical mechanism lies under the quantum probe that grants its advantage, in the language of open quantum systems.
We emphasize that the tempararal correlation history of $M$, through $P-M$ enatanglement, can flow into $P$, a reminiscent of quantum non-Markovianity. 



\section[3]{Quantum probe: learning decoherence beyond response theory}
% This should be a secton introducing/reviewing the power of quamtum-probe-based learning: we will focus on a single-qubit dephasing case, and through some calculation, 
% one shows the decoherence and unitary error are related to the anti-commutator & commutator respective -- showing that response theory can't reveal probe decoherence
% and Matter's anti-commutator; using nontrivial quantum probe is essential. 

We now demonstrate explicitly how a quantum probe reveals information that is fundamentally inaccessible to system-only response theory. Focusing on the paradigmatic case of single-qubit dephasing, we show that probe decoherence and probe phase shifts are governed by distinct operator sectors of the matter: the anti-commutator and commutator correlators, respectively. This separation makes precise why classical response theory cannot characterize decoherence or fluctuations, and why a nontrivial quantum probe is essential.

Consider a qubit probe coupled longitudinally to a matter operator \(M(t)\),
\begin{equation}
H_I(t)=\frac{1}{2}\,y(t)\,\sigma_z\otimes M(t),
\end{equation}
where \(y(t)\) is a known control modulation applied to the probe and \(M(t)\) is a Hermitian operator in the Heisenberg picture with respect to the matter Hamiltonian. We assume an initially factorized state \(\rho(0)=\rho_P\otimes\rho_M\).

Because \(\sigma_z\) has eigenvalues \(\pm 1\), the interaction induces conditional matter evolutions,
\begin{equation}
U_I(T)=|0\rangle\!\langle0|\otimes U_+(T)+|1\rangle\!\langle1|\otimes U_-(T),
\end{equation}
with
\begin{equation}
U_\pm(T)=\mathcal{T}\exp\!\left(\mp\frac{i}{2\hbar}\int_0^T dt\,y(t)M(t)\right).
\end{equation}

The probe coherence evolves as (\AGENT{ $L$ should be $\mathcal{L}$})
\begin{equation}
\rho_{01}(T)=\rho_{01}(0)\,L(T),
\qquad
L(T)=\mathrm{Tr}_M\!\left[U_+(T)\rho_M U_-^\dagger(T)\right].
\end{equation}
The complex-valued functional \(L(T)\) fully characterizes both decoherence and coherent phase accumulation of the probe.


Define the matter correlators
\begin{equation}
\mathcal{M}^+(t,t')=\langle\{M(t),M(t')\}\rangle/2,
\qquad
\mathcal{M}^-(t,t')=\langle[M(t),M(t')]\rangle/2,
\end{equation}
with \(\langle\cdot\rangle=\mathrm{Tr}_M(\rho_M\cdot)\).
For Hermitian \(M(t)\), \(\mathcal{M}^+(t,t')\) is real while \(\mathcal{M}^-(t,t')\) is purely imaginary.

Expanding the influence functional to second order (or equivalently within the Gaussian cumulant approximation), one obtains
\begin{equation}
\ln L(T)=-\chi(T)+i\phi(T),
\end{equation}
where
\begin{equation}
\chi(T)\propto
\int_0^T dt\!\int_0^T dt'\,y(t)y(t')\,\mathcal{M}^+(t,t'),
\end{equation}
and
\begin{equation}
\phi(T)\propto
\int_0^T dt\!\int_0^t dt'\,y(t)y(t')\,\frac{1}{i}\mathcal{M}^-(t,t').
\end{equation}

This result has a direct physical interpretation:
\begin{itemize}
\item The real functional \(\chi(T)\) governs the decay of probe coherence, \(|L(T)|=e^{-\chi(T)}\), and depends solely on the anti-commutator correlator \(\mathcal{M}^+\), which encodes \textit{fluctuations} and noise power of the matter.
\item The phase \(\phi(T)\) depends solely on the commutator correlator \(\mathcal{M}^-\), corresponding to the retarded response of the matter and producing a coherent, unitary phase shift of the probe (\textit{dissipation}).
\end{itemize}

Why response theory fails? \WD{this paragraph is by AI; and its analysis is untested. }
This separation highlights a fundamental limitation of classical response theory. System-only response functions are constructed exclusively from commutators and thus probe only the causal response sector of the matter. They cannot access \(\mathcal{M}^+(t,t')\), and therefore cannot characterize decoherence, noise strength, or fluctuation-induced irreversibility.
By contrast, the quantum probe evolves conditionally along forward and backward branches, and its reduced dynamics necessarily involves both commutator and anti-commutator correlators. Decoherence is not a response of the matter; it is a manifestation of matter fluctuations rendered operationally visible through entanglement with the probe.
In this sense, quantum-probe-based learning goes beyond response theory not by improving sensitivity, but by accessing an algebraic sector—symmetrized correlators—that is entirely invisible to classical perturb-and-measure protocols.

\WD{Some thoughts: (1) fluctuation(decoherence) is absent in closed-system dynamics hence undetectable, while quatum-probe formalism is open-quantum system-based. (2) the fluctuation of $M$ is essentially non-Markovian and is detectable in $P$ after registery.   }


\section[4]{Quantum Probe: learning non-equilibriant Gaussian many-body systems}
% Switch to talk about probe advantage for many-body system. For the Gaussian case, the anti-commutator can be inferred from commutator (math express) through FDT if the system is thermal equilibrium(?).
% While in general, DFT does not appiled and a plethoral of interesting physical properpites remain inaccessible through response theory.
We now move from few-level probes sensing an abstract ``bath operator'' to the regime where the matter itself is an extended many-body system. (\AGENT{language needs improvement })
The key message is that, for Gaussian matter (free bosons/fermions, or more generally quadratic Hamiltonians and Gaussian states), probe access to
symmetrized correlators (anti-commutators) enables reconstruction of physical quantities that are not accessible from response (commutator) data alone.
In thermal equilibrium the two sectors are related by the fluctuation--dissipation theorem (FDT), so in that special case fluctuation information can
be inferred from response. Out of equilibrium, the KMS/FDT relation fails and the anti-commutator sector carries independent information (e.g.\ occupations),
which is invisible to response theory.

Throughout this section, we assume the probe can be programmed to couple to a chosen matter operator (or an operator basis on a chosen spatial region),
so that the probe signal reconstructs the corresponding two-point correlators. In the Gaussian setting, two-point data already determine a wide class
of state properties.

\paragraph{Equilibrium caveat (FDT).}
For a stationary thermal state $\rho \propto e^{-\beta H}$ and a Hermitian matter operator $M(t)$,
define the symmetrized correlator and retarded susceptibility
\begin{equation}
\mathcal{M}^+(\omega)\equiv \int_{-\infty}^{\infty} d\tau\, e^{i\omega\tau}\,\frac{1}{2}\langle\{M(\tau),M(0)\}\rangle,
\qquad
\chi^R(\tau)\equiv \frac{i}{\hbar}\Theta(\tau)\langle[M(\tau),M(0)]\rangle,
\end{equation}
with $\chi^R(\omega)=\int d\tau\,e^{i\omega\tau}\chi^R(\tau)$.
Thermal equilibrium (KMS) implies the FDT relation
\begin{equation}
\mathcal{M}^+(\omega)=\hbar\,\coth\!\Big(\frac{\beta\hbar\omega}{2}\Big)\,\Im\chi^R(\omega).
\end{equation}
Hence, in equilibrium one may infer symmetrized fluctuations from commutator/response data.
Outside equilibrium (general stationary diagonal ensembles, GGEs, quenched states), $\mathcal{M}^+(\omega)$ and $\chi^R(\omega)$ are not functionally linked:
learning $\mathcal{M}^+$ is a genuinely new capability enabled by probe decoherence.

\subsection[3_a]{Learning occupation number of Bosonic systems}
% For Bosos with Gaussian statistics, provide a conrete recipe (detailed math) for occupation number reconstruction from anti-commutators
We consider a bosonic matter system with a quadratic Hamiltonian
\begin{equation}
H_M = \sum_k \hbar \omega_k\, b_k^\dagger b_k ,
\end{equation}
and a probe--matter coupling of the form
\begin{equation}
H_I(t) = h(t)\otimes M(t), \qquad
M(t) = \sum_k g_k \bigl( b_k e^{-i\omega_k t} + b_k^\dagger e^{i\omega_k t} \bigr),
\end{equation}
where \(h(t)\) is a controlled probe operator.

For a number-conserving, Gaussian bosonic state with mode occupations
\(\langle b_k^\dagger b_{k'} \rangle = n_k \delta_{k k'}\),
the symmetrized correlator is (\AGENT{need to verify the following !!})
\begin{equation}
\mathcal{M}^+(t,t') \equiv \langle \{ M(t), M(t') \} \rangle/2
= \sum_k |g_k|^2 \bigl( 2 n_k + 1 \bigr)\cos\!\bigl[\omega_k (t-t')\bigr].
\end{equation}

In probe-based spectroscopy, this correlator appears directly in the probe decoherence functional. 
Because \(\mathcal{M}^+(t,t')\) depends explicitly on the mode occupations \(n_k\), full knowledge of the symmetrized correlator in time or frequency space allows direct reconstruction of \(\{n_k\}\), provided the couplings \(g_k\) and mode frequencies \(\omega_k\) are known.

By contrast, system-only response theory accesses only the commutator correlator
\begin{equation}
\mathcal{M}^-(t,t') = \langle [M(t),M(t')] \rangle/2
= i \sum_k |g_k|^2 \sin\!\bigl[\omega_k (t-t')\bigr],
\end{equation}
which is independent of \(n_k\). Thus, outside thermal equilibrium, occupation numbers are fundamentally invisible to response theory.


\WD{MATH details to verify the above two expressions is elsewhere (YES). So for non-thermal case where DFT is nonapplicable, occupation number estimation is a smoking gun of probe advantage. }

\subsection[3_b]{Learning entanglement entropy for non-interacting systems}
% Now switch to non Newman entropy, focus on a susbsystem A, imagine through probe, one can reconstruct anticommutator & commutator of operators applied on any subsystem ;
% then show how to obtained the non Newman entropy explicitly (detailed math) for bosons and fermions respectively.

We now turn to von Neumann entanglement entropy.
For generic interacting systems, entanglement entropy is not an ordinary real-time correlator functional and generally requires replica or multi-copy
information. However, for \emph{non-interacting (Gaussian) bosons and fermions}, the reduced state on a subsystem is Gaussian/quasi-free, and its
entropy is determined entirely by two-point correlators restricted to that subsystem. Thus, if probe measurements can reconstruct the relevant
(commutator/anti-commutator) two-point functions of a tomographically complete operator set on a region $A$, then $S(\rho_A)$ is explicitly obtainable.

\subsubsection{Entanglement entropy reconstruction for free bosons}

\paragraph{Covariance matrix from anti-commutators.}
Let $N$ bosonic modes have canonical quadrature vector
\begin{equation}
\hat{R}=(\hat{x}_1,\hat{p}_1,\ldots,\hat{x}_N,\hat{p}_N)^{T},
\qquad
[\hat{R}_i,\hat{R}_j]=i\hbar\,\Omega_{ij},
\end{equation}
with symplectic form $\Omega=\bigoplus_{m=1}^{N}\begin{pmatrix}0&1\\-1&0\end{pmatrix}$.
Define the (symmetrized) covariance matrix
\begin{equation}
V_{ij} \equiv \frac{1}{2}\langle\{\Delta \hat{R}_i,\Delta \hat{R}_j\}\rangle,
\qquad
\Delta \hat{R}_i\equiv \hat{R}_i-\langle \hat{R}_i\rangle.
\label{eq:boson_cov_def}
\end{equation}
Thus, equal-time anti-commutator correlators of quadratures (or any equivalent complete linear combination) 
$\mathcal{M}^{+}=\langle \{\Delta \hat{R}(t), \Delta \hat{R}(t')\} \rangle|_{t\rightarrow t'}$ determine $V$.

Let $A$ be a subsystem consisting of $N_A$ modes.
Denote by $V_A$ and $\Omega_A$ the restrictions of $V$ and $\Omega$ to those modes.

\paragraph{Symplectic spectrum.}
By Williamson's theorem, there exists a symplectic matrix $S_A$ such that
\begin{equation}
V_A = S_A\left(\bigoplus_{\alpha=1}^{N_A}\nu_\alpha\,\mathbb{I}_2\right)S_A^{T},
\end{equation}
where $\nu_\alpha\ge \hbar/2$ are the symplectic eigenvalues of $V_A$.
Operationally, they are obtained as the positive eigenvalues of $|i\Omega_A V_A|$:
\begin{equation}
\mathrm{eig}\big(i\Omega_A V_A\big)=\{\pm \nu_\alpha\}_{\alpha=1}^{N_A}.
\label{eq:symplectic_eigs}
\end{equation}

\paragraph{Entropy formula.}
For a Gaussian reduced state $\rho_A$, the von Neumann entropy is a sum over bosonic thermal entropies of the normal modes:
\begin{equation}
S(\rho_A)=\sum_{\alpha=1}^{N_A}
\left[
\left(\bar{n}_\alpha+1\right)\ln\left(\bar{n}_\alpha+1\right)
-\bar{n}_\alpha\ln \bar{n}_\alpha
\right],
\qquad
\bar{n}_\alpha=\frac{\nu_\alpha}{\hbar}-\frac{1}{2}.
\label{eq:boson_entropy}
\end{equation}
Combining Eqs.~\eqref{eq:boson_cov_def}--\eqref{eq:boson_entropy} yields an explicit reconstruction chain:
\[
\{\text{equal-time anti-commutators on }A\}\;\Rightarrow\;V_A\;\Rightarrow\;\{\nu_\alpha\}\;\Rightarrow\;S(\rho_A).
\]


\subsubsection{Entanglement entropy reconstruction for free fermions: bilinear probes and restored probe advantage}

We now revisit entanglement entropy reconstruction for non-interacting fermionic systems, and show that a genuine probe advantage can be restored once the probe couples to \emph{bosonic bilinears} of fermionic operators, rather than to linear (Majorana) modes.

\paragraph{Setup and entropy formula.}
Consider a quadratic fermionic system and a spatial subsystem $A$. For any Gaussian fermionic state, the reduced density matrix $\rho_A$ is fully determined by the one-body correlation matrix
\begin{equation}
C_{ij} \equiv \langle f_i^\dagger f_j \rangle , \qquad i,j \in A .
\end{equation}
The entanglement entropy is given by
\begin{equation}
S_A = -\mathrm{Tr}\!\left[ C \ln C + (I-C)\ln(I-C) \right],
\end{equation}
or equivalently by the eigenvalues $\{\lambda_\alpha\}$ of $C$.
Thus, learning $S_A$ is equivalent to reconstructing the fermionic covariance matrix $C$.

\paragraph{Failure of linear fermionic probes.}
If the probe couples linearly to fermionic operators (e.g.\ Majoranas $\gamma_a$), then the symmetrized correlator
\begin{equation}
\frac{1}{2}\langle \{\gamma_a(t),\gamma_b(t')\} \rangle = \delta_{ab}
\end{equation}
is fixed by the canonical anticommutation relations and contains no state information. All information relevant for $C$ resides in the commutator sector, which can in principle be accessed through response-type measurements. In this restricted setting, probe-based access does not yield an information advantage.

\paragraph{Bilinear fermionic observables.}
To overcome this obstruction, we instead consider probe couplings to \emph{fermionic bilinears}, which are bosonic operators and physically accessible via density- or current-type couplings. Specifically, we define
\begin{equation}
M(t) = \sum_{i,j\in A} h_{ij}\, f_i^\dagger(t) f_j(t),
\end{equation}
where $h=h^\dagger$ specifies a chosen spatial and orbital pattern within subsystem $A$.
Such operators naturally arise as local densities, currents, or mode-resolved occupations.

\paragraph{Noise correlator and state dependence.}
We consider the symmetrized correlator
\begin{equation}
\mathcal{M}^+(t,t') \equiv \frac{1}{2}\langle \{M(t),M(t')\} \rangle .
\end{equation}
Since $M$ is bilinear in fermions, $\mathcal{M}^+$ is a four-fermion correlator. For Gaussian states, Wick’s theorem yields
\begin{equation}
\mathcal{M}^+(t,t')
=
\sum_{ijkl} h_{ij} h_{kl}
\Big[
\langle f_i^\dagger(t) f_l(t') \rangle
\langle f_j(t) f_k^\dagger(t') \rangle
-
\langle f_i^\dagger(t) f_k^\dagger(t') \rangle
\langle f_j(t) f_l(t') \rangle
\Big].
\end{equation}
In number-conserving Gaussian states, anomalous terms vanish, and the symmetrized correlator depends quadratically on the one-body correlator $C_{ij}$.
Thus, $\mathcal{M}^+$ carries explicit information about fermionic occupations and coherences within $A$.

\paragraph{Probe-induced decoherence and access to $\mathcal{M}^+$.}
We now couple a controllable qubit probe via a pure-dephasing interaction
\begin{equation}
H_{\mathrm{int}} = \lambda\, \sigma_z \otimes M .
\end{equation}
Under standard cumulant or filter-function treatments, the probe coherence obeys
\begin{equation}
\ln L(T) = - \lambda^2 \int \frac{d\omega}{2\pi}\,
|F(\omega;T)|^2\, \mathcal{M}^+(\omega)
+ i(\cdots),
\end{equation}
where $F(\omega;T)$ is a probe control filter.
Hence, probe decoherence directly measures the symmetrized correlator $\mathcal{M}^+$, i.e.\ the \emph{fermionic noise} associated with bilinear observables.

\paragraph{Why response alone is insufficient.}
System-only linear response experiments access the retarded susceptibility
\begin{equation}
\chi^R_{MM}(t) = -i\theta(t)\langle [M(t),M(0)] \rangle,
\end{equation}
which depends only on commutators. Out of thermal equilibrium, the retarded response does \emph{not} determine the symmetrized correlator $\mathcal{M}^+$, nor the underlying covariance matrix $C$, in the absence of fluctuation–dissipation relations.
Thus, occupation information — and hence entanglement entropy — is not identifiable from response data alone.

\paragraph{Restored probe advantage.}
By designing multiple bilinear observables $M^{(\alpha)}$ (different $h^{(\alpha)}$ patterns) and reconstructing their noise spectra $\mathcal{M}^{+(\alpha)}(\omega)$ via probe decoherence, one can tomographically recover the fermionic covariance matrix $C_{ij}$ within subsystem $A$. This enables reconstruction of the full entanglement entropy $S_A$.

Therefore, even for free fermions, probe-based measurements provide access to state information (the Keldysh/noise sector) that is fundamentally inaccessible to system-response measurements alone. This restores a genuine and operational probe advantage for fermionic entanglement spectroscopy beyond equilibrium settings.



\section[5]{Quantum Probe: learning entanglement entropy of interacting system}
%%%%%%%%%%%%%%%%%%%%%%%%%%%%%%%%%%%%%%%%%%%%%%%%%%%%%%%%%%%%%%%%%%%%%%%%%%%%%%

In Sec.~IV-B we exploited Gaussianity: for free bosons/fermions the reduced state $\rho_A$ is Gaussian/quasi-free,
hence $S(\rho_A)$ is an explicit functional of two-point data restricted to $A$.
For genuinely interacting matter, $\rho_A$ is non-Gaussian (\WD{subsystem or full system non-Gaussian?}) and $S(\rho_A)=-\Tr(\rho_A\ln\rho_A)$ is a nonlinear
functional of $\rho_A$. Nevertheless, if the interacting reduced state is close (in a controlled sense) to a Gaussian
reference $\rho_{A,G}$, one can develop a perturbative expansion of $S(\rho_A)$ whose  nontrivial corrections
are governed by higher connected correlators, beginning at the four-point level. In this section we make this statement
precise and turn it into a constructive recipe: (i) learn the Gaussian reference $\rho_{A,G}$ from probe-accessible
two-point data on $A$, (ii) learn the connected four-point cumulants on $A$, and (iii) contract them through a
\emph{Gaussian modular kernel} fixed by $\rho_{A,G}$ to obtain the leading interacting correction to $S(\rho_A)$. [\WD{Might be able to exend to higher-order correction. }]

%------------------------------------------------------------------------------
\subsection{Entropy expansion around a Gaussian reference}
%------------------------------------------------------------------------------

Let the reduced state on $A$ (all is on  system $M$ and hence drop for briefty) be written as
\begin{equation}
\rho_A=\rho_{A,G}+\delta\rho_A,\qquad \Tr_A(\delta\rho_A)=0,
\label{eq:rho_expand}
\end{equation}
where $\rho_{A,G}$ is a Gaussian/quasi-free state on $A$ fixed by its one- and two-point functions, and
\begin{equation}
K_{A,G}:=-\ln\rho_{A,G}
\label{eq:K_G_def}
\end{equation}
is the corresponding Gaussian modular Hamiltonian (\WD{realte to entanglement Hamiltonian?}).

A rigorous expansion follows from the Fr\'echet derivative of the operator logarithm. For a full-rank density operator
$\rho$ and a small perturbation $X$, one has
\begin{equation}
\ln(\rho+X)
=
\ln\rho
+\int_{0}^{\infty}ds\,(\rho+s)^{-1}X(\rho+s)^{-1}
-\frac12\int_{0}^{\infty}ds\int_{0}^{\infty}ds'\,
(\rho+s)^{-1}X(\rho+s')^{-1}X(\rho+s)^{-1}
+O(X^3).
\label{eq:log_frechet}
\end{equation}
Applying this to $\rho=\rho_{A,G}$ and $X=\delta\rho_A$, and expanding $S(\rho_A)=-\Tr[\rho_A\ln\rho_A]$,
the first-order correction is the entanglement ``first law'',
\begin{equation}
\delta S_A^{(1)}=\Tr\!\big(\delta\rho_A\,K_{A,G}\big)
= -\Tr\!\big(\delta\rho_A\ln\rho_{A,G}\big).
\label{eq:ent_first_law}
\end{equation}

At second order, the apparently ``two'' quadratic contributions that arise from naively multiplying
$\rho_{A,G}+\delta\rho_A$ by the series \eqref{eq:log_frechet} recombine into a single compact quadratic form:
\begin{equation}
\delta S_A^{(2)}= -\frac12\int_{0}^{\infty}ds\;
\Tr\!\left[ \delta\rho_A\,(\rho_{A,G}+s)^{-1}\,\delta\rho_A\,(\rho_{A,G}+s)^{-1}  \right].
\label{eq:deltaS2_resolvent}
\end{equation}
(\WD{CAUTION: I didn't manage to replicate same $\delta S^{(1,2)}_A$}!) Equivalently, introducing the Bogoliubov--Kubo--Mori (BKM) [\WD{research}] super-operator
\begin{equation}
\mathcal{J}_{\rho}^{-1}(X):=\int_{0}^{\infty}ds\,(\rho+s)^{-1}X(\rho+s)^{-1},
\label{eq:BKM_def}
\end{equation}
one may write
\begin{equation}
\delta S_A^{(2)}
=
-\frac12\,\Tr\!\left[\delta\rho_A\,\mathcal{J}_{\rho_{A,G}}^{-1}(\delta\rho_A)\right].
\label{eq:deltaS2_BKM}
\end{equation}
In the eigenbasis $\rho_{A,G}=\sum_n p_n|n\rangle\langle n|$, this becomes
\begin{equation}
\delta S_A^{(2)}
=
-\frac12\sum_{m,n}
\frac{|\langle m|\delta\rho_A|n\rangle|^2}{p_m+p_n},
\label{eq:deltaS2_eigbasis}
\end{equation}
which makes concavity manifest: $\delta S_A^{(2)}\le 0$.

\paragraph{Why the first-order term can be made to vanish.}
If $\rho_{A,G}$ is chosen as the \emph{maximum-entropy Gaussian/quasi-free} state consistent with the
\emph{exact} one- and two-point data on $A$, then $\delta S_A^{(1)}$ vanishes.
Indeed, $K_{A,G}$ is quadratic in the canonical operators on $A$ (bosons: quadratures; fermions: bilinears),
so $\Tr(\delta\rho_A K_{A,G})$ depends only on deviations of one- and two-point functions.
By construction these deviations are zero, hence
\begin{equation}
\delta S_A^{(1)}=0,
\qquad
S(\rho_A)=S(\rho_{A,G})+\delta S_A^{(2)}+O(\delta\rho_A^3).
\label{eq:first_order_zero}
\end{equation}
This is the controlled sense in which the leading \emph{interacting} correction is ``four-point dominated''.

%------------------------------------------------------------------------------
\subsection{Gaussian normal ordering, connected four-point cumulants, and modular flow}
%------------------------------------------------------------------------------

Let $R=(R_1,\dots,R_{2|A|})$ denote a complete set of canonical operators on $A$:
for bosons one may take quadratures $R=(\hat x_1,\hat p_1,\dots)$, and for fermions one may take Majoranas
(or equivalently $c^\dagger c$ bilinears for number-conserving states).
Given $\rho_{A,G}$, define Gaussian (Wick) normal ordering $:\cdots:_G$ (\WD{cumulant representation: connected correlation functions // maybe change notation? }) by subtracting all contractions computed
with the Gaussian two-point function. For instance,
\begin{equation}
:R_iR_j:_G:=R_iR_j-\langle R_iR_j\rangle_G,
\label{eq:normal_order_2}
\end{equation}
and analogously for quartics by subtracting the three Wick pairings so that $\langle :R_iR_jR_kR_l:_G\rangle_G=0$.

The connected four-point cumulant tensor on $A$ is
\begin{equation}
\kappa_{ijkl}
:=
\langle R_iR_jR_kR_l\rangle_{\rho_A}
-
\sum_{\text{pairings}}
\langle R_aR_b\rangle_{\rho_A}\,\langle R_cR_d\rangle_{\rho_A}.
\label{eq:kappa_def}
\end{equation}
For a Gaussian/quasi-free state $\kappa_{ijkl}\equiv 0$; thus $\kappa$ is the first stable signature of
non-Gaussianity beyond covariance data.

To connect $\kappa$ to $\delta S_A^{(2)}$ constructively, expand the true modular Hamiltonian
$K_A:=-\ln\rho_A$ around the Gaussian modular Hamiltonian $K_{A,G}$:
\begin{equation}
K_A
=
K_{A,G}+\delta K_A,
\qquad
\delta K_A
=
\frac{1}{4!}\sum_{ijkl}g_{ijkl}\,:R_iR_jR_kR_l:_G
+O(R^6).
\label{eq:deltaK_quartic}
\end{equation}
The induced linear change of the state is given by the \WD{Duhamel formula}
\begin{equation}
\delta\rho_A
=
-\int_0^1 d\tau\;
\rho_{A,G}^{\,1-\tau}\,\delta K_A\,\rho_{A,G}^{\,\tau}
\;+\;
\rho_{A,G}\,\Tr(\rho_{A,G}\delta K_A)
+O(\delta K_A^2).
\label{eq:duhamel_delta_rho}
\end{equation}
Define Gaussian modular flow
\begin{equation}
O(\tau):=\rho_{A,G}^{\,\tau}O\rho_{A,G}^{-\tau},\qquad 0\le \tau\le 1.
\label{eq:modular_flow_def}
\end{equation}
Substituting \eqref{eq:duhamel_delta_rho} into \eqref{eq:deltaS2_BKM} yields the quadratic form
\begin{equation}
\delta S_A^{(2)}
=
-\frac{1}{2(4!)^2}
\sum_{ijkl}\sum_{mnpq}
g_{ijkl}\;\mathcal{K}^{(G)}_{ijkl,mnpq}\;g_{mnpq}
+O(g^3),
\label{eq:deltaS2_g_kernel}
\end{equation}
with the \emph{Gaussian modular kernel}
\begin{equation}
\mathcal{K}^{(G)}_{ijkl,mnpq}
:=
\int_0^1 d\tau\int_0^1 d\tau'\;
\Big\langle
\big(:R_iR_jR_kR_l:_G\big)(\tau)\;
\big(:R_mR_nR_pR_q:_G\big)(\tau')
\Big\rangle_G.
\label{eq:K_kernel_def}
\end{equation}

Finally, the measurable connected cumulant $\kappa$ is linearly related to $g$ at weak non-Gaussianity:
\begin{equation}
\kappa_{ijkl}
=
-\sum_{mnpq}\chi^{(4,G)}_{ijkl,mnpq}\,g_{mnpq}+O(g^2),
\label{eq:kappa_g_linear}
\end{equation}
where $\chi^{(4,G)}$ is a Gaussian quartic susceptibility computable from $\rho_{A,G}$ (Wick + modular flow).
Inverting \eqref{eq:kappa_g_linear} (on the symmetry-allowed subspace) gives $g\sim (\chi^{(4,G)})^{-1}\kappa$ and
\begin{equation}
\delta S_A^{(2)}
=
-\frac12
\sum_{ijkl}\sum_{mnpq}
\kappa_{ijkl}\;\widetilde{\mathcal{K}}^{(G)}_{ijkl,mnpq}\;\kappa_{mnpq}
+O(\kappa^3),
\label{eq:deltaS2_kappa_kernel}
\end{equation}
for an explicit kernel $\widetilde{\mathcal{K}}^{(G)}$ determined entirely by the Gaussian reference
(i.e.\ by two-point data on $A$) and the inversion of $\chi^{(4,G)}$.

\paragraph{Operational meaning.}
Equation \eqref{eq:deltaS2_kappa_kernel} is the precise content of ``entropy-from-multipoint correlators'' in the
weakly interacting regime: the leading interacting correction to entanglement entropy is a \emph{bilinear functional}
of the connected four-point sector on $A$, filtered by a kernel fixed by the Gaussian reference.
Thus, if a probe protocol reconstructs (i) the two-point data on $A$ (to determine $\rho_{A,G}$) and (ii) the connected
four-point cumulants on $A$, then the leading correction $\delta S_A^{(2)}$ is operationally accessible.
By contrast, system-only response theory (nested commutators) does not determine the symmetrized/connected four-point
sector out of equilibrium, so it does not, in general, suffice to compute \eqref{eq:deltaS2_kappa_kernel}.

%------------------------------------------------------------------------------
\subsection{Example I: weakly interacting bosons ($\phi^4$)}
%------------------------------------------------------------------------------

\subsubsection{Model and Gaussian reference}
Consider an interacting bosonic field theory (continuum, for definiteness)
\begin{equation}
H \;=\; H_0 + \lambda V,
\qquad
H_0=\frac12\int d^dx\;\Big[\pi^2+(\nabla\phi)^2+m^2\phi^2\Big],
\qquad
V=\int d^dx\;\frac{1}{4!}\phi^4(x),
\label{eq:phi4_model}
\end{equation}
with $[\phi(x),\pi(y)]=i\hbar\delta(x-y)$. Fix a region $A$ and denote the reduced state by
$\rho_A=\Tr_{\bar A}\rho$. For $\lambda\neq 0$, $\rho_A$ is non-Gaussian.

Choose $\rho_{A,G}$ as the maximum-entropy Gaussian state matching the exact one- and two-point data of
$(\phi,\pi)$ on $A$:
\begin{equation}
\langle R_i\rangle_{\rho_A}=\langle R_i\rangle_{\rho_{A,G}},\qquad
\langle R_iR_j\rangle_{\rho_A}=\langle R_iR_j\rangle_{\rho_{A,G}},
\qquad
R\equiv(\phi(x\in A),\pi(x\in A)).
\label{eq:phi4_gaussian_match}
\end{equation}
Then the first-order correction vanishes, $\delta S_A^{(1)}=0$, and
\begin{equation}
S(\rho_A)=S(\rho_{A,G})+\delta S_A^{(2)}+O(\lambda^3),
\label{eq:phi4_entropy_expand}
\end{equation}
with $\delta S_A^{(2)}$ given by the universal formulas \eqref{eq:deltaS2_resolvent}--\eqref{eq:deltaS2_kappa_kernel}.

\subsubsection{Connected four-point data as the leading interacting signature}
Define the connected four-point cumulant on $A$,
\begin{equation}
\kappa_{ijkl}
=
\langle R_iR_jR_kR_l\rangle_{\rho_A}
-
\sum_{\text{pairings}}
\langle R_aR_b\rangle_{\rho_A}\,\langle R_cR_d\rangle_{\rho_A}.
\label{eq:phi4_kappa_def}
\end{equation}
For a Gaussian state, $\kappa\equiv 0$. In the weakly interacting $\phi^4$ theory, $\kappa=O(\lambda)$:
diagrammatically, $\kappa$ is generated by the first connected four-leg vertex correction beyond Wick factorization.

\subsubsection{Entropy correction as a quadratic functional of $\kappa$}
Expanding the modular Hamiltonian as in \eqref{eq:deltaK_quartic},
\begin{equation}
\delta K_A
=
\frac{1}{4!}\sum_{ijkl}g_{ijkl}\,:R_iR_jR_kR_l:_G + O(R^6),
\label{eq:phi4_deltaK}
\end{equation}
one obtains
\begin{equation}
\delta S_A^{(2)}
=
-\frac{1}{2(4!)^2}
\sum_{ijkl}\sum_{mnpq}
g_{ijkl}\;\mathcal{K}^{(G)}_{ijkl,mnpq}\;g_{mnpq}
+O(g^3),
\label{eq:phi4_deltaS2_g}
\end{equation}
with $\mathcal{K}^{(G)}$ given by \eqref{eq:K_kernel_def}. Using the linear relation between $g$ and $\kappa$,
Eq.~\eqref{eq:kappa_g_linear}, we may rewrite this as
\begin{equation}
\delta S_A^{(2)}
=
-\frac12
\sum_{ijkl}\sum_{mnpq}
\kappa_{ijkl}\;\widetilde{\mathcal{K}}^{(G)}_{ijkl,mnpq}\;\kappa_{mnpq}
+O(\kappa^3).
\label{eq:phi4_deltaS2_kappa}
\end{equation}
Thus, to leading order in the interaction, learning the connected four-point sector on $A$ (together with the
two-point sector fixing $\rho_{A,G}$) suffices to determine the leading interacting correction to entanglement entropy.

\paragraph{Probe-accessible content.}
In bosonic many-body platforms, probe dephasing spectroscopy accesses symmetrized (anti-commutator) sectors of
multi-time correlators. Reconstructing the equal-time cumulant tensor $\kappa$ on $A$ therefore requires access to
the non-Gaussian part of four-point (and mixed $\phi$--$\pi$) correlators, which is generically invisible to system-only
response theory out of equilibrium.

%------------------------------------------------------------------------------
\subsection{Example II: weak-$U$ Hubbard fermions}
%------------------------------------------------------------------------------

\subsubsection{Model and quasi-free Gaussian reference}
Consider the Hubbard model on a lattice,
\begin{equation}
H = H_0 + U V,
\qquad
H_0= -t\sum_{\langle ij\rangle,\sigma}(c_{i\sigma}^\dagger c_{j\sigma}+\mathrm{h.c.})
-\mu\sum_{i,\sigma} n_{i\sigma},
\qquad
V=\sum_i n_{i\uparrow}n_{i\downarrow},
\label{eq:hubbard_model}
\end{equation}
with $n_{i\sigma}=c_{i\sigma}^\dagger c_{i\sigma}$. Fix a subsystem $A$ (set of sites). For $U=0$ the reduced state
$\rho_A$ is quasi-free and its entropy is determined by the correlation matrix
\begin{equation}
(C_A)_{i\sigma,j\sigma'}=\langle c_{i\sigma}^\dagger c_{j\sigma'}\rangle,
\qquad
S(\rho_A)=-\sum_\alpha\Big[\lambda_\alpha\ln\lambda_\alpha+(1-\lambda_\alpha)\ln(1-\lambda_\alpha)\Big],
\quad \{\lambda_\alpha\}=\mathrm{eig}(C_A).
\label{eq:fermion_entropy_free}
\end{equation}

For weak $U\neq 0$, define $\rho_{A,G}$ to be the maximum-entropy quasi-free state matching the exact two-point
function on $A$:
\begin{equation}
\langle c_{i\sigma}^\dagger c_{j\sigma'}\rangle_{\rho_A}
=
\langle c_{i\sigma}^\dagger c_{j\sigma'}\rangle_{\rho_{A,G}}
\quad \forall\, i,j\in A.
\label{eq:hubbard_gaussian_match}
\end{equation}
Then $K_{A,G}=-\ln\rho_{A,G}$ is quadratic in fermions,
\begin{equation}
K_{A,G}=\sum_{i,j\in A}\sum_{\sigma,\sigma'} h^{(A)}_{i\sigma,j\sigma'}\,c_{i\sigma}^\dagger c_{j\sigma'}+\mathrm{const.},
\label{eq:fermion_KG_quad}
\end{equation}
and again $\delta S_A^{(1)}=\Tr(\delta\rho_A K_{A,G})=0$ by construction.

\subsubsection{Non-Gaussian datum: connected four-fermion cumulants}
For quasi-free fermions, Wick's theorem holds and all higher correlators reduce to two-point contractions.
Interactions manifest as \emph{Wick violations} captured by connected four-fermion cumulants. A general connected
tensor is
\begin{align}
\kappa_{i\sigma,j\sigma',k\tau,l\tau'}
:=
&\;
\langle c_{i\sigma}^\dagger c_{j\sigma'}^\dagger c_{k\tau} c_{l\tau'}\rangle_{\rho_A}
\nonumber\\
&-
\Big(
\langle c_{i\sigma}^\dagger c_{l\tau'}\rangle_{\rho_A}\,\langle c_{j\sigma'}^\dagger c_{k\tau}\rangle_{\rho_A}
-
\langle c_{i\sigma}^\dagger c_{k\tau}\rangle_{\rho_A}\,\langle c_{j\sigma'}^\dagger c_{l\tau'}\rangle_{\rho_A}
\Big).
\label{eq:fermion_kappa_def}
\end{align}
A commonly used reduced set is the connected density--density correlator on $A$,
\begin{equation}
\kappa^{(n)}_{i,j}
:=
\langle n_i n_j\rangle_{\rho_A}-\langle n_i\rangle_{\rho_A}\langle n_j\rangle_{\rho_A},
\qquad
n_i:=\sum_\sigma c_{i\sigma}^\dagger c_{i\sigma},
\label{eq:dens_dens_connected}
\end{equation}
but for generic entanglement corrections the full tensor \eqref{eq:fermion_kappa_def} is the natural object.
For small $U$, $\kappa=O(U)$.

\subsubsection{Quartic modular correction and entropy correction}
Expand the true modular Hamiltonian $K_A$ around the quasi-free $K_{A,G}$ by the leading quartic term,
\begin{equation}
K_A
=
K_{A,G}
+\delta K_A,
\qquad
\delta K_A
=
\frac{1}{4}\sum_{i,j,k,l\in A}\sum_{\sigma,\sigma',\tau,\tau'}
g_{i\sigma,j\sigma',k\tau,l\tau'}\;
:\!c_{i\sigma}^\dagger c_{j\sigma'}^\dagger c_{k\tau} c_{l\tau'}\!:\_G
+\cdots,
\label{eq:fermion_deltaK_quartic}
\end{equation}
where $:\cdots:\_G$ denotes quasi-free normal ordering with respect to $\rho_{A,G}$.

Using the same Duhamel expansion and BKM form as above, one obtains
\begin{equation}
\delta S_A^{(2)}
=
-\frac12
\sum_{\alpha,\beta}
g_{\alpha}\;\mathcal{K}^{(G)}_{\alpha\beta}\;g_{\beta}
+O(g^3),
\label{eq:fermion_deltaS2_g}
\end{equation}
where $\alpha,\beta$ are composite indices labeling quartic monomials
$\mathcal{O}_\alpha=:c^\dagger c^\dagger c c:\_G$, and
\begin{equation}
\mathcal{K}^{(G)}_{\alpha\beta}
=
\int_0^1 d\tau\int_0^1 d\tau'\;
\langle \mathcal{O}_\alpha(\tau)\,\mathcal{O}_\beta(\tau')\rangle_G,
\qquad
\mathcal{O}(\tau)=\rho_{A,G}^{\,\tau}\mathcal{O}\rho_{A,G}^{-\tau}.
\label{eq:fermion_Kkernel}
\end{equation}
As in the bosonic case, the connected four-fermion cumulant \eqref{eq:fermion_kappa_def} is linearly related to $g$
at weak coupling,
\begin{equation}
\kappa_{\alpha} = -\sum_{\beta}\chi^{(4,G)}_{\alpha\beta}\,g_\beta + O(g^2),
\label{eq:fermion_kappa_g}
\end{equation}
and therefore
\begin{equation}
\delta S_A^{(2)}
=
-\frac12
\sum_{\alpha,\beta}
\kappa_{\alpha}\;\widetilde{\mathcal{K}}^{(G)}_{\alpha\beta}\;\kappa_{\beta}
+O(\kappa^3),
\label{eq:fermion_deltaS2_kappa}
\end{equation}
with $\widetilde{\mathcal{K}}^{(G)}$ determined entirely by the quasi-free reference (hence by $C_A$).

\paragraph{What is special about fermions.}
Equal-time anti-commutators are fixed by the fermionic algebra, so state information at two-point level is already
fully encoded in $C_A=\langle c^\dagger c\rangle$. The probe advantage here is therefore not ``anti-commutator vs
commutator'' at two-point level, but the operational access to non-Gaussian four-point sectors (Wick violations)
entering \eqref{eq:fermion_deltaS2_kappa}. Out of equilibrium, system-only response theory does not determine these
connected four-point tensors, whereas probe-based protocols can access them through higher-order correlators in the
reduced probe dynamics.

%%%%%%%%%%%%%%%%%%%%%%%%%%%%%%%%%%%%%%%%%%%%%%%%%%%%%%%%%%%%%%%%%%%%%%%%%%%%%%
% End of Section V
%%%%%%%%%%%%%%%%%%%%%%%%%%%%%%%%%%%%%%%%%%%%%%%%%%%%%%%%%%%%%%%%%%%%%%%%%%%%%%


\section{Conclusion}




% ----------------------------------------------------------------------
% Bibliography placeholder (optional)
% \bibliography{refs}
% ----------------------------------------------------------------------

\end{document}
